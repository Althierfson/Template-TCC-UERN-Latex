% ------------------------------------------------------------------------
% TCC_CC:   Modelo de Trabalho Monográfico Acadêmico UERN 
%           para o curso de Bacharelado em Ciência da Computação
%
% Esse template é baseado no template 
% desenvolvido por Francisco Reinaldo
% Disponivel em: https://pt.overleaf.com/latex/templates/utfpr-modelo-tcc-monografia-dissertacao-tese-com-pdf-a-3b-versao-goldendragon/znzmpctjxmqq)
%
% Sendo modificado para se encaixar 
% nos requisitos solicitados pela UERN no ano de 2020
%
% Caso você tenha caído de cabeça e não 
% entenda nada de overleaf, peça ajuda a
% quem indicou :P
%
% Ou der seus pulos e aprenda sozinho.
% Dá um google ai.
%
% Mais para dar um colher de chá eu coloquei alguns tutoriais :)
%
% A maioria dos dados (fora os texto em si)
% que você precisar modificar em relação ao o seu nome,
% orientador, data, banca estão no arquivo 
% dados-gerais dá uma olhada lá. Mas fique a 
% vontade para editar como quiser :).
%
% Desenvolvido por: Althierfson Tullio O aluno
%
% Versao de Controle: v32 - Codinome Externo: GoldenDragon
%
% ------------------------------------------------------------------------


% --- LianTze / Reinaldo´s Fine Tuning
\RequirePackage{scrlfile}
\AfterClass{memoir}{\usepackage[a-3b]{pdfx}} %v.3
%Para funcionar, deixe na raiz: sRGB_IEC61966-2-1_black_scaled
%\BeforePackage{hyperref}{\usepackage[a-3b]{pdfx}} %v.3.2
% ---

\documentclass[
	% -- opções da classe memoir --
	12pt,				% tamanho da fonte
	oneside,			% para impressão em frente. Oposto a twoside (frente,costas)
	a4paper,			% tamanho do papel. 
	%sumario=tradicional,
	% % -- opções da classe abntex2 --
	chapter=TITLE, % títulos de capítulos convertidos em letras maiúsculas
	%section=title,
	% -- opções do pacote babel --
	english,			% idioma adicional para hifenização
	brazil				% o último idioma é o principal do documento
	]{abntex2}

% --- 
% PACOTES PARA AJUSTE e CORREÇÕES EM ABNTEX2
% --- 
\input{pacotes_oficiais}
% ---

\usepackage{fancyhdr}
%\usepackage[margin=1.5cm]{caption}

% --- 
% DADOS BÁSICOS DE AUTORIA
% --- 
%---------EXCLUSIVA ENTRADA DE DADOS PELO ALUNO-------

% Sei que aqui tem muito coisas, isso é culpa do cara que eu copiei esse templates
% muitos você não precisará mexer.

% Veja com calma e você rapidamente entender, se você for de computação, imagine que esse campo é um monte de variáveis, porquer é o que é.

\def \titulodissertacao{Projeto de uma arma de portais dimensionais}
\def \discentenome{Rick}
\def \discentesobrenome{Sanchez}
\def \discente{\discentenome~\discentesobrenome}
\def \RAouMatricula{1234567}
\def \defesadatacompletacomdiadasemana{\_\_\_\_/\_\_\_\_/\_\_\_\_}
\def \defesahora{19h30min}
\def \defesasala{Q204}
\def \datadeentregadotrabalhoparaleitura{29 de Fevereiro de 2020}

\def \proforientador{Prof. Dr. Albert Einstein}
\def \proforientadorInstituicao{Universidade do Estado do Rio Grande do Norte - UERN}

\def \profcoorientador{Viajante 1302}
\def \profcoorientadorInstituicao{Federação intergaláctica 6568}

\def \profbancaA{Prof. Dr. Elon Musk}
\def \profbancaAInstituicao{Universidade do Estado do Rio Grande do Norte - UERN}

\def \profbancaB{Prof. Dr. Tony Stark}
\def \profbancaBInstituicao{Universidade do Estado do Rio Grande do Norte - UERN}

\def \email{rick@cidadela.com}
\def \telefonecomDDD{(46) 666-666}
\def \cpf{1101.010.10011-1100001}
\def \rg{1101.010.10011-1100001}

\def \restricaopublicacao{Não Restringir}% Escolha em: Total , Parcial , Não Restringir
\def \restricaoparcialtexto{}
\def \restricaoTotal{Não há.}
\def \restricaoParcial{Não há.}
\def \agenciafomento{Não há.}
\def \cidadededefesa{Natal, Rio Grande do Norte}
%------------------------------------------------------
\def \tipotrabalhoescrito{TCC (Graduação - Bacharelado em Ciência da Computação)}
\def \instituicaotrabalho{Universidade do Estado do Rio Grande do Norte}

\def \campus{Campus Avançado de Natal}
 
% --- 
% CAPA
% --- 
\renewcommand{\imprimircapa}{%
  \begin{capa}%
    \center
    \parbox{4cm} {\includegraphics[scale=0.37]{0-capa-e-rosto/MIVAtivo-3MARCA.png}}\\
    \vspace{0.3 cm}
    \ABNTEXchapterfont \bfseries\MakeUppercase {\instituicaotrabalho}
    \par
    CAMPUS AVANÇADO DE NATAL
    \par
    DEPARTAMENTO DE CIÊNCIA DA COMPUTAÇÃO \\
    BACHARELADO EM CIÊNCIA DA COMPUTAÇÃO     
    
    {\vspace*{3cm} \ABNTEXchapterfont\large\MakeUppercase\imprimirautor}

    \vspace*{4cm}
    \begin{center}
    \ABNTEXchapterfont\bfseries\MakeUppercase\imprimirtitulo
    \vspace*{3em}
    \end{center}
    \vfill
    
    \MakeUppercase\imprimirlocal

    \imprimirdata
    
    \vspace*{1cm}
  \end{capa}
} 
 
% --- 
% FOLHA DE ROSTO
% --- 
\preambulo{%
Monografia apresentada á Universidade do Estado do Rio Grande do Norte - UERN - como requisito obrigatório para obtenção do titulo de Bacharelado em Ciência da Computação.
}

% --- 
% DADOS BÁSICOS DE AUTORIA EM ABNTEX, LOAD DE dados-gerais.tex
% --- 
\tipotrabalho{\tipotrabalhoescrito}
\instituicao{UNIVERSIDADE DO ESTADO DO RIO GRANDE DO NORTE}
\autor{\discente} 
\titulo{\titulodissertacao} 
\data{2020} % <-- Ano do trabalho
\local{NATAL} % <-- localidade

\orientador{\proforientador } % <-- Nome do orientador
\coorientador{\profcoorientador} % <-- Nome do Co-Orientador (Opcional)

% ---
% COMPILA O INDICE
% ---
\makeindex


\begin{document}
% Seleciona o idioma do documento (conforme pacotes do babel)
\selectlanguage{brazil}
% ----------------------------------------------------------
% ELEMENTOS PRÉ-TEXTUAIS
% ----------------------------------------------------------
% \pretextual

% ---
% Capa
% ---
\imprimircapa

% ---
% Folha de rosto
% ---
%\imprimirfolhaderosto

    \newpage
    \thispagestyle{empty}
    
    \begin{center}
    	\MakeUppercase\discente\\
    	\vspace{4cm}
    	\MakeUppercase\titulodissertacao
    \end{center}
    	\vspace{4cm}
    \begin{flushright}
    \begin{minipage}{8cm}
    \imprimirpreambulo
    
    \vspace{0.5cm}
    Orientador: \proforientador
    %CO-ORIENTADOR :
    
    \end{minipage}
    \end{flushright}
     
    \vspace{7cm}
    
    
    \begin{center}
    \MakeUppercase{Natal} \\
    2020
    \end{center}


% imprimir ficha catalográfica PDF
% \includepdf[pages={-}]{0-capa-e-rosto/FichaCatalografica.pdf}
% ---
% Inserir folha de aprovação
% ---
% Isto é um exemplo de Folha de aprovação, elemento obrigatório da NBR
% 14724/2011 (seção 4.2.1.3). Você pode utilizar este modelo até a aprovação
% do trabalho. Após isso, substitua todo o conteúdo deste arquivo por uma
% imagem da página assinada pela banca com o comando abaixo:
%
%\includepdf{1-pre-textuais/FolhaAprovacaoAssinada.pdf}


%ajusta tamanho linha textual horizontal se necessário
\setlength{\ABNTEXsignwidth}{15cm} 

%coloque 1pt se desejar linha. Comumente, linha de assinatura é utilizada para pessoas não letradas.
\setlength{\ABNTEXsignthickness}{1pt} 

%espaçamento entre assinaturas
\setlength{\ABNTEXsignskip}{1.5cm} 


\begin{folhadeaprovacao}
	\begin{center} 
		\ABNTEXchapterfont\MakeUppercase\imprimirautor

        \vspace*{\fill}
        \begin{center}
         	\ABNTEXchapterfont\MakeUppercase\imprimirtitulo
        \end{center}
        \vspace*{\fill}

		\hspace{.45\textwidth} 
		\begin{minipage}{.5\textwidth}
			\imprimirpreambulo
		\end{minipage}
		\vspace*{\fill}
    \end{center}
    
    Aprovado em~\defesadatacompletacomdiadasemana.
    \vspace*{\fill}
    
%\the\year.
    
    \begin{center}
        Banca examinadora
    \end{center}
    
	\assinatura{\proforientador~(Orientador)\\{\proforientadorInstituicao}}
	%\assinatura{\imprimircoorientador\\{\footnotesize \profcoorientadortitulacao\\(Co-orientador UTFPR)}} 
    
    %\assinatura{\profpresidentebanca\\{\footnotesize \profpresidentebancatitulacao\\(Presidente da Banca UTFPR)}} 
    \assinatura{\profbancaA\\{\profbancaAInstituicao}}
    \assinatura{\profbancaB\\{\profbancaBInstituicao}}


	\vspace*{\fill}
	%\begin{center}
	%\noindent Folha de Aprovação assinada encontra-se arquivada na Coordenação do %Curso.
	%\end{center} 
\end{folhadeaprovacao} 

% ---
% Dedicatória
% ---
%NBR 6029:2006: 3.11 dedicatória: Texto em que o(s) autor(es) presta(m) homenagem e/ou dedica(m) seu trabalho.

\begin{dedicatoria}
	\vspace*{\fill}
	\begin{flushright}
		Dedico este trabalho a mim mesmo, \\
        porque sou muito foda.
	\end{flushright}
\end{dedicatoria}

% ---
% Agradecimentos
% ---
\input{1-pre-textuais/agradecimentos}

% ---
% Epígrafe
% ---
%NBR 14724:2011 3.14: epígrafe: texto em que o autor apresenta uma citação, seguida de indicação de autoria, relacionada com a matéria tratada no corpo do trabalho

\begin{epigrafe}
	\vspace*{\fill}
	\begin{flushright}
		\textit{``Não acho que quem ganhar ou quem\\
		perder, nem quem ganhar nem perder, \\
		vai ganhar ou perder. Vai todo mundo perder.``\\
		(Dilma Rousseff)
		}
	\end{flushright}
\end{epigrafe}


% ---
% RESUMOS
% ---

% Resumo em português
%NBR 6028: de 150 a 500 palavras os de trabalhos acadêmicos (teses, dissertações e outros) e relatórios técnico-cientifícos;
\def \palavraschaves{Palavra-Chave1. Palavra-Chave2. Palavra-Chave3. Palavra-Chave4.}
\begin{resumo}[\protect\bfseries Resumo]  
Aqui entra o resumo do seu TCC.

\vspace{\onelineskip} 

\noindent \textbf{Palavras-chave}: \palavraschaves
\end{resumo}

% Resumo em inglês
\def \palavraschavesIngles{palavra-chave1. palavra-chave2. palavra-chave3. palavra-chave4.}
\begin{resumo}[\protect\bfseries Abstract] 
\begin{otherlanguage*}{english}
Seu resumo em Inglês, ou em outro idioma.

\vspace{\onelineskip} 

\noindent \textbf{Keywords}: \palavraschavesIngles 

\end{otherlanguage*} 
\end{resumo} 

% ---
% Lista de ilustrações
% ---
\pdfbookmark[0]{\listfigurename}{lof}
\listoffigures*

\cleardoublepage
% ---

% ---
% Lista de tabelas
% ---
\pdfbookmark[0]{\listtablename}{lot}
\listoftables*
\cleardoublepage
% ---

% ---
% inserir lista de quadros
% ---
%\pdfbookmark[0]{\listofquadrosname}{loq}
%\listofquadros*
%\cleardoublepage
% ---

% ---
% Lista de Abreviaturas e Siglas
% ---
%\input{1-pre-textuais/lista-de-abreviaturas} 

% ---
% Lista de Símbolos
% ---
%\input{1-pre-textuais/lista-de-simbolos}

% ---
% Lista de Codigo-Fonte
% ---
%\pdfbookmark[0]{\listofcodigosname}{loc}
%\listofcodigos*
%\cleardoublepage
% ---

% ---
% Sumario
% ---
\pdfbookmark[0]{\contentsname}{toc}
\tableofcontents*
\cleardoublepage



% ----------------------------------------------------------
% ELEMENTOS TEXTUAIS
% ----------------------------------------------------------
\textual

\pagestyle{fancy}
\fancyhf{}
\rhead{ \small{\thepage}}
\renewcommand{\headrulewidth}{0pt}

\sffamily
\chapter{INTRODUÇÃO}
\label{Introdução}

Aqui você pode fazer o sua introdução.
\chapter{FUNDAMENTAÇÃO TEÓRICA}
\label{Fundamentacao}


Aqui você pode escrever sua fundamentação teórica.
\chapter{TRABALHOS RELACIONADOS}
\label{Relacionados}

Aqui você pode escrever a seção de trabalhos relacionados.

Aqui temos um exemplo de como adicionar uma figurar (vaja a figura \ref{fig_Exemplo}), para que ele fique de acordo com as normas exigidas pela UERN. Por motivos de organização as figuras são salvas na pasta ''2-textuais/figs/''.

% Copie e cole esse código abaixo

\begin{figure}[!ht]
    \centering
    \begin{minipage}{11cm}
    \centering
    \caption{Rick Sanchez.}
    \fbox{\includegraphics[width=10cm]{2-textuais/figs/Rick.jpg}}
    \label{fig_Exemplo}
    \caption*{Fonte: Elaborada pelo autor.}
    \end{minipage}
\end{figure}

% -------

% Descrição dos comandos:

% \caption{Titulo da sua figura}
% \fbox{\includegraphics[width=Mude o tamanho da figura]{Local/Caminho da sua figura}}
%   OBS:caso vc mude o valor do campo Width, certifique-se de mudar o valor do campo
%       \Begin`{minipage{mude aquie}} para um acima do valor da figura, tipo 14-15, 13-14,         etc.
% \label{Qualquer nome}, use isso para poder identificar as figuras no Texto como no exemplo acima.

% \caption*{Legenda da figura}
\chapter{METODOLOGIA}
\label{Metodologia}

Aqui você pode escrever a metodologia do seu trabalho.
\chapter{DESENVOLVIMENTO}
\label{Desenvolvimento}

Aqui você pode escrever o desenvolvimento do seu trabalho.

Aqui tem um exemplo de como adicionar uma tabela. Tabela no latex é um lance complicado, então recomendo o site: \textbf{''https://www.tablesgenerator.com/''}, ira facilitar bastante.

Recomendo que as tabelas sejam montadas em arquivo .tex específicos e salvos na pasta ''2-textuais/Tabelas/'', mas você pode colocar no meu do seu texto.

Veja no exemplo abaixo como usar um tabela (Tabela \ref{tabela_par_impar}) (Essa tabela estar salva a pasta Tabelas)

\begin{table}[!ht]
\centering
\begin{minipage}{8cm}

    \centering
    \caption{Números pares e impares até 10}
    \label{tabela_par_impar}
    \begin{tabular}{l|l}
    \hline
    \textbf{Números Par} & \textbf{Números Impar} \\ \hline
    2 & 1 \\ \hline
    4 & 3 \\ \hline
    6 & 5 \\ \hline
    8 & 7 \\ \hline
    10 & 9 \\ \hline
    \end{tabular}
    
\end{minipage}
\end{table}

% Para alinhar o título da tabela com as margem, não tem jeito, procurei uma forma mas só encontrei isso.

% o segredo é o \begin{minipage}{8cm} e o \end{minipage}, sempre coloque sua tabela entres esse dois argumentos, você terá que alterar o valor 8 até encontrar o ponto que fique alinhado

%Para referenciar aqui no latex é muito simples, você só precisa do seguinte comando: "\cite{identificador da referencia}", veja o exemplo abaixo

Aqui temos um exemplo da referencia mais comum: \cite{Paradeda2019you}. As referencia são guardadas no arquivo \textbf{''referencias-acervo.bib''}, de uma olhada nesse arquivo se você não entende muito do latex.

Esse simples comando já será suficiente para fazer a referencia no texto e também coloca-la la na seção de referencias. Tudo de forma automática.

Existi outros forma de referencia, colocar só o nome dos autores, a data do trabalho, e por ai vai, da um pesquisada que você acha fácil.
\chapter{\bfseries  CONSIDERAÇÃO FINAL E CONCLUSÃO}
\label{Conclusao}

Aqui entrar sua conclusão.

Aqui também tem uma forma de referenciar um Apêndice, veja o exemplo: "Aqui temos o projeto da arma de portais (Apêndice \ref{AP:Q_ExpreFaciais})"". Para o anexo é a mesma coisa.
%\input{2-textuais/4-descricao_sistema.tex}
%\input{2-textuais/6-resultados}

% ----------------------------------------------------------
% Finaliza a parte no bookmark do PDF
% para que se inicie o bookmark na raiz
% e adiciona espaço de parte no Sumário
% ----------------------------------------------------------
\phantompart

% ---
% Conclusão
% ---
%\input{2-textuais/5-conclusao}

% ---
% Trabalhos Futuros
% ---
%\input{2-textuais/7-trabs-futuros}


% ----------------------------------------------------------
% ELEMENTOS PÓS-TEXTUAIS
% ----------------------------------------------------------
\postextual

% ---
% Referências
% ---
%\newpage
%\renewcommand{\thepage}{}
\bibliography{referencias-acervo.bib}

%\begin{comment}
% ---
% Glossário
% ---
\input{3-pos-textuais/glossario}

% ---
% Apêndices
% ---
\begin{apendicesenv}

% Imprime uma página indicando o início dos apêndices
\partapendices
%\addcontentsline{toc}{section}{Seção Não Numerada}
%NBR 6029:2006: 3.3 apêndice: Texto ou documento elaborado pelo autor, a fim de complementar sua argumentação, sem prejuízo da unidade nuclear do trabalho. 

%-------------------------------\usepackage{pdfpages}       
% Para Incluir PDFs---------------------------

% ----------------------------------------------------------
\chapter{Projeto da arma de portais}
% ----------------------------------------------------------
\label{AP:Q_ExpreFaciais} % <- Para você identificar no texto.

Esquema de como ficará a arma de portais.

\includepdf[scale=0.8, pages={-}, pagecommand={}]{3-pos-textuais/apendices/ProjeotDaArma.pdf}

% para adicionar o arquivo (que precisar ser PDF) use o comando abaixo, mudando apenas o local do arquivo.

% \includepdf[scale=0.8, pages={-}, pagecommand={}]{Local do arquivo}


\end{apendicesenv}


% ---
% Anexos
% ---
\begin{comment}
\begin{anexosenv}

% Imprime uma página indicando o início dos anexos
\partanexos

\input{3-pos-textuais/anexos}

\end{anexosenv}
\end{comment}
% ---
% INDICE REMISSIVO
% ---
\phantompart
\printindex


%---------------------------------------------------------------------
%\end{comment}
\end{document}