\chapter{TRABALHOS RELACIONADOS}
\label{Relacionados}

Aqui você pode escrever a seção de trabalhos relacionados.

Aqui temos um exemplo de como adicionar uma figurar (vaja a figura \ref{fig_Exemplo}), para que ele fique de acordo com as normas exigidas pela UERN. Por motivos de organização as figuras são salvas na pasta ''2-textuais/figs/''.

% Copie e cole esse código abaixo

\begin{figure}[!ht]
    \centering
    \begin{minipage}{11cm}
    \centering
    \caption{Rick Sanchez.}
    \fbox{\includegraphics[width=10cm]{2-textuais/figs/Rick.jpg}}
    \label{fig_Exemplo}
    \caption*{Fonte: Elaborada pelo autor.}
    \end{minipage}
\end{figure}

% -------

% Descrição dos comandos:

% \caption{Titulo da sua figura}
% \fbox{\includegraphics[width=Mude o tamanho da figura]{Local/Caminho da sua figura}}
%   OBS:caso vc mude o valor do campo Width, certifique-se de mudar o valor do campo
%       \Begin`{minipage{mude aquie}} para um acima do valor da figura, tipo 14-15, 13-14,         etc.
% \label{Qualquer nome}, use isso para poder identificar as figuras no Texto como no exemplo acima.

% \caption*{Legenda da figura}